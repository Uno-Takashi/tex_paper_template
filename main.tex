%------------------------------------------------------------------------------
\documentclass[master]{settings/cimt}
% オプションについては,マニュアルを参照.
% \documentclass[master,oneside]{cimt} 

% 必要とするパッケージがあれば、以下のファイルで読み込む。
\usepackage{settings/packages}

% 論文タイトル
\jtitle{日本語タイトル}
% 長い時には自動的に改行されるが,次のように明示的に改行することもできる.
% \jtitle{長いタイトルを\\このように改行位置を指定して組む}

% 英文タイトル
\etitle{English title}

% 著者名
\jauthor{東工大太郎}

% 英文著者名
\eauthor{Tokodai Tarou}

% 指導教員
\supervisor{教員名}

% 提出月.この例だと,2010年1月.
\handin{2025}{1}


\begin{document}

% 表紙と表紙裏
\maketitle 

% ここから前文
\frontmatter

% 概要
\begin{jabstract}
\input contents/abstract_japanese.tex
\end{jabstract}

% 英文概要
\begin{eabstract}
\input contents/abstract_english.tex
\end{eabstract}

% 目次
\setcounter{tocdepth}{2}
\tableofcontents

% ここから本文
\mainmatter

\input contents/chapter1.tex

% ここから後付
\backmatter

% 発表文献
% \pubUseLongName % 指定すると,タイトルが 「発表文献と研究活動」になる.
% \begin{publications}
% \input contents/publications.tex
% \end{publications}

% 参考文献: BibTeX を使う場合の例 (styleは適宜選択)
\setcitestyle{numbers}
\bibliographystyle{plainnat}
\begingroup
\let\em\upshape
\bibliography{bibliography}
\endgroup
% https://zenn.dev/grafumo/scraps/9daded9a870f2d

% 参考文献: 直接記述する場合の例
% \input biblio.tex

% 謝辞 (前文においても良い)
\begin{acknowledgements}
\input contents/acknowledgement.tex
\end{acknowledgements}

%付録 (必要な場合のみ)
\appendix

\input contents/appendix.tex

\end{document}
